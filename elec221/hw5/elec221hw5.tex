\documentclass[11pt, fleqn]{article}
\usepackage[utf8]{inputenc}
\usepackage{fullpage}
\usepackage{amsmath,amssymb,array}
\usepackage{dsfont}
\usepackage{amsfonts}
\usepackage{graphicx}
\usepackage{mathtools}
\usepackage{polynom}
\usepackage{esint}
\usepackage{mathrsfs}
\usepackage{calligra}
\usepackage{style}
\usepackage{graphicx}
\graphicspath{ {./images/} }
\setlength{\parindent}{0em}



\title{ELEC 221 HW5}
\author{Kyle Mackenzie}
\date{November 10, 2022}

\begin{document}
\allowdisplaybreaks

\maketitle

\pagebreak

\section*{HW5.6: Signal Reframing}
Consider an ideal discrete-time lowpass filter with impulse response and for which the frequency response is that shown in the figure below. Let us consider obtaining a new filter with impulse response $ h_1[n]$ and frequency response $ H_1(e^{j\w}) $ as follows:

$$
    h_1[n] = \begin{cases}
        h[n/2], & \text{n even} \\
        0, & \text{n odd}
    \end{cases}
$$


This corresponds to inserting a sequence value of zero between each sequence value of $ h[n] $. Determine and sketch $ H_1(e^{j\w}) $ and state the class of ideal filters to which it belongs (e.g., lowpass, highpass, bandpass, multiband, etc.) 

\image[Filt.PNG]{Ideal discrete-time l;lowpass Filter}

\subsection*{HW5.6 Solution}

This filter is essentially an upsampled version of the original filter with width $ \w_c $, where we are sampling twice as frequent as with the original signal.

Thus, for $ H_1(e^{j\w}) $, we can expect a compression by a factor of 2 in the frequency domain, and a periodic extension about $ \pi $, since the sampling rate has been doubled.

\image[h1.PNG]{Frequency Response of new filter}

This filter is a multiband filter, as it allows bands $-\frac{\w_c}{2} < \w < \frac{\w_c}{2}$ and $\pi-\frac{\w_c}{2} < \w < \pi + \frac{\w_c}{2}$ to pass.

\pagebreak

\section*{HW5.7: Interpolation}

The figure shows a system consisting of a continuous-time linear time-invariant system followed by a sampler, conversion to a sequence, and a discrete-time linear time-invariant system. The continuous-time LTI system is causal and satisfies the LCCDE:

$$ \frac{dy_c(t)}{dt} + y_c(t) = x_c(t) $$

\image[q5.7.PNG]{System}

The input
is a unit impulse $\delta (t)$ 

(a) Determine $ y_c(t) $ 

(b) Determine the frequency response $H(\Omega)$ and the impulse $h[n]$ such that $w[n]=\delta [n].$ 

Upload a pdf or an image.

\subsection*{HW5.7 Solution}

%% 5.7 a)
\subsection*{a)}

This LCCDE corresponds to a frequency response of 
$$ 
    H_{LTI}(j\w) = \frac{1}{1 + j\w}
$$

so, we have 
\begin{align*}
    &Y(j\w) = X(j\w) H_{LTI}(j\w)  \\
    &Y(j\w) = \mathcal{F}(\delta(t))H_{LTI}(j\w)  \\
    &Y(j\w) = H_{LTI}(j\w) \\
    &\Rightarrow y_c(t) = \mathcal{F}^{-1}(H_{LTI}(j \w)) \\ 
    &y_c(t) = \mathcal{F}^{-1}( \frac{1}{1 + j\w}) \\ 
    &y_c(t) = e^{-t}u(t) \text{, using TABLE 4.2 BASIC FOURIER TRANSFORM PAIRS from Oppenheim p.328}
\end{align*}

%% 5.7 b)
\subsection*{b)}

We want to find frequency response $H(\Omega)$ and corresponding impulse response $h[n]$ such that $w[n]=\delta [n]$.

Looking at Figure 3, we take $ y_c(t) $ and multiply it by $p(t) = \sum_{n=-\infty}^{+\infty} \delta(t - nT) $.

In the frequency domain, we have:
\begin{align*}
    &y[n] = y_c(nT) \\
    &y[n] = e^{-nT}u(nT) \\
    &Y(e^{j\w}) = \mathcal{F}(e^{-nT}u(nT)) \\
    &\text{using   } a^nu[n], \hspace*{1mm} |a|<1 \hspace*{1mm} \rightarrow \hspace*{1mm} \frac{1}{1 - ae^{-j\w}} \\
    &\text{Take } a^n = e^{-nT} = (e^{-T})^n \\
    &\Rightarrow Y(e^{j\w}) = \mathcal{F}(e^{-nT}u(nT)) = \frac{1}{1 - e^{-T}e^{-j\w}}
\end{align*}

And now we have the spectra of the input into the second LTI system. We want to find $H(\Omega)$ such that:
$$
    Y(e^{j\w})H(\Omega) = W(\Omega)
$$
where
$$
W(\Omega) \xrightarrow{\mathcal{F}^{-1}} \delta[n]
$$
\begin{align*}
    & \Rightarrow W(\Omega) = 1 = Y(e^{j\w})H(\Omega) = \frac{H(\Omega)}{1 - e^{-T}e^{-j\w}} \\
    &\Rightarrow H(\Omega) = 1 - e^{-T}e^{-j\w} \\
    &\Rightarrow h[n] = \delta[n] - e^{-T}\delta[n-1]
\end{align*}

\vspace*{2mm}

So we have $H(\Omega) = 1 - e^{-T}e^{-j\w}$ and $h[n] = \delta[n] - e^{-T}\delta[n-1]$.

\pagebreak

\section*{HW5.8: Sampling}

Consider the system in Figure 1

\image[q5.81.PNG]{}

Given the Fourier transform of $ x_r(t) $
in Figure 4, sketch the Fourier transform of two different signals $ x(t) $ that could have generated $ x_r(t) $:

\image[q5.82.PNG]{}

\subsection*{HW5.8 Solution}

This system is functionally a system that samples a signal, turns it into a DT sequence, and then applies a lowpass filter at $ \w_c = W = \frac{\pi}{T_s} = \frac{1}{2}\w_s$ in order to remove the periodic extensions of sampling function $P(j\w) = \frac{2\pi}{T}\sum_{k=-\infty}^{+\infty}\delta(\w - k\w_s) $

\hspace*{1mm}

The effects of aliasing tell us that any frequency components present in the input signal that exist in the range

$$
    \frac{\w_s}{2} < |\w| < \w_s
$$

will get mirrored about $ \frac{\w_s}{2} $ in the frequency domain.

\hspace*{1mm}

Therefore, we could have two signals we'll call $x_1(t)$ and $x_2(t)$ with spectra $X_1(j\w)$ and $X_2(j\w)$ that end up as the same result due to aliasing where the first one has the same exact spectra as $x_r(t)$ and the second has the spectra of $x_r(t)$ mirrored about $\w = \frac{\w_s}{2}$.

that will have the same fourier spectra as $ x_r(t) $, once passed through the sampling system.

\bigimage[X1.PNG]{Frequency spectra of $X_1(j\w)$}
\bigimage[X2.PNG]{Frequency spectra of $X_2(j\w)$}

NOTE: The peak in the diagrams is of height 1.
\end{document}
